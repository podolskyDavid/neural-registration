\chapter{\abstractname}

This thesis explores an innovative approach to intraoperative brain registration by utilizing Neural Radiance Fields (NeRFs) as differentiable, implicit representations of brain surface geometry and appearance. Unlike conventional mesh-based techniques, NeRFs allow for direct optimization of camera positions via backpropagation, significantly enhancing alignment accuracy between preoperative and intraoperative imaging. We introduce a robust, model-agnostic implementation of neural registration within the nerfstudio framework, overcoming previous limitations regarding customization and adaptability. The primary scientific contribution involves a comprehensive analysis of multiple loss functions—including L1, L2, Structural Similarity Index (SSIM), Normalized Cross-Correlation (NCC), and Mutual Information (MI)—to assess their impact on registration accuracy, convergence rate, and stability. Experimental results demonstrate that while L1 loss offers rapid and stable convergence, MI and NCC are notably resilient to NeRF-generated visual artifacts, providing insights crucial for clinical applications. By advancing NeRF-based registration techniques, this work contributes directly toward improving the precision and reliability of image-guided neurosurgical procedures.


\makeatletter
\ifthenelse{\pdf@strcmp{\languagename}{english}=0}
{\renewcommand{\abstractname}{Kurzfassung}}
{\renewcommand{\abstractname}{Abstract}}
\makeatother

\chapter{\abstractname}

\begin{otherlanguage}{ngerman}
    Diese Bachelorarbeit untersucht einen innovativen Ansatz zur intraoperativen Hirnregistrierung durch die Verwendung von Neural Radiance Fields (NeRFs) als differenzierbare, implizite Darstellungen der Hirnoberflächen-Geometrie und -Erscheinung. Im Gegensatz zu konventionellen mesh-basierten Techniken ermöglichen NeRFs eine direkte Optimierung der Kamerapositionen mittels Backpropagation, was die Genauigkeit der Ausrichtung zwischen präoperativer und intraoperativer Bildgebung erheblich verbessert. Wir stellen eine robuste, modellunabhängige Implementierung der neuralen Registrierung innerhalb des Nerfstudio-Frameworks vor, die bisherige Einschränkungen hinsichtlich Anpassungsfähigkeit und Flexibilität überwindet. Der primäre wissenschaftliche Beitrag umfasst eine umfassende Analyse verschiedener Verlustfunktionen—einschließlich L1, L2, Structural Similarity Index (SSIM), Normalized Cross-Correlation (NCC) und Mutual Information (MI)—um deren Einfluss auf die Registrierungsgenauigkeit, Konvergenzrate und Stabilität zu bewerten. Experimentelle Ergebnisse zeigen, dass während der L1-Verlust eine schnelle und stabile Konvergenz bietet, MI und NCC bemerkenswert widerstandsfähig gegenüber von NeRF erzeugten visuellen Artefakten sind, was Erkenntnisse liefert, die für klinische Anwendungen entscheidend sind. Durch die Weiterentwicklung NeRF-basierter Registrierungstechniken trägt diese Arbeit direkt zur Verbesserung der Präzision und Zuverlässigkeit bildgeführter neurochirurgischer Eingriffe bei.
\end{otherlanguage}


% Undo the name switch
\makeatletter
\ifthenelse{\pdf@strcmp{\languagename}{english}=0}
{\renewcommand{\abstractname}{Abstract}}
{\renewcommand{\abstractname}{Kurzfassung}}
\makeatother