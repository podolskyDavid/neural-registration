% !TeX root = ../main.tex
% Add the above to each chapter to make compiling the PDF easier in some editors.

\chapter{Introduction}\label{chapter:introduction}

Neurosurgery is a high-precision medical field where accuracy directly influences patient outcomes. During brain tumor resections, surgeons rely on image-guided navigation systems to assist with spatial orientation and accurately locate critical anatomical structures \parencite{navab2015surgical}. These systems typically use a process called registration, which refers to aligning preoperative Magnetic Resonance Imaging (MRI) data with the patient's anatomy \parencite{Alam2017Medical}. 

The field of computer vision has recently seen remarkable advancements in neural scene representation techniques, particularly with the introduction of Neural Radiance Fields (NeRFs). NeRFs represent scenes as continuous functions that map 3D coordinates and viewing directions to color and density values, enabling high-quality 3D depictions of brain anatomical surfaces. These implicit neural representations have revolutionized how we model and render 3D environments, offering differentiable, continuous scene representations that can be optimized through gradient-based methods \parencite{mildenhall2020nerf}.

Real-world example of an integrated intraoperative system is Advanced Multimodality Image Guided Operating (AMIGO) Suite which is being heavily used in surveries in Brigham and Women's Hospital.

This thesis builds upon recent work that leverages NeRFs for pose estimation \parencite{yen2020inerf} and, more specifically, for intraoperative registration \parencite{fehrentz2024intraoperative}. 
% We propose to enhance NeRF-based intraoperative registration through two main contributions: (1) a comprehensive exploration of alternative loss functions beyond the standard L2 loss, and (2) an investigation into the effects of style transfer on registration accuracy.

We propose to enhance NeRF-based intraoperative registration through a comprehensive exploration of alternative loss functions beyond the standard L2 loss.

Such research can be highly relevant for AR/VR-assisted surgeries. \cite{Cho2020Enhancing}

Current golden standards of intraoperative registration techniques face several challenges:

\begin{enumerate}
    \item \textbf{Brain shift}: The brain's position and shape change during surgery due to cerebrospinal fluid drainage, gravity, surgical manipulations, and other physical, surgical, or biological changes, impacting the precision of surgical interventions. \cite{Iversen2018Automatic}
    \item \textbf{Surgical workflow integration}: Registration methods should integrate seamlessly into existing surgical workflows without requiring additional equipment or extensive time. MRI, while used for preoperative diagnostic purposes, in-surgery require specific O.R. design, significant workflow interruption, and specific non-magnetic equipment. \cite{Gandhe2018Intraoperative}
%    \item \textbf{Cross-modal alignment}: Matching preoperative MRI data with intraoperative camera images remains difficult.
    \item \textbf{Speed and accuracy}: Registration must be both precise and fast enough to be clinically viable during surgery. \cite{Riva2020Intraoperative}
\end{enumerate}

The approach presented by \textcite{fehrentz2024intraoperative} addresses these challenges by using neural rendering for registration. However, their work primarily focuses on a hypernetwork-based approach for appearance adaptation and employs a standard L2 loss for optimization. This thesis extends their work by exploring whether alternative loss functions might yield better registration results and by analyzing how different style transfer techniques affect the registration process.

% \section{Research Questions}


\section{Research Questions and Objectives}\label{section:research-questions}

This thesis aims to address the following primary research question:

\begin{itemize}
    \item \textbf{How do different loss functions — specifically L1, L2, Structural Similarity Index, Normalized Cross-Correlation, and Mutual Information — affect the convergence speed and accuracy of NeRF-based intraoperative registration?}
\end{itemize}

To comprehensively investigate this question, we address several sub-questions:

\begin{enumerate}
    \item How do various loss functions differ in their convergence rates and final registration accuracy when aligning preoperative NeRF renderings with intraoperative images?
    
    \item What are the computational efficiency trade-offs between different loss functions in time-sensitive surgical environments?
    
    \item Which loss functions demonstrate greater robustness to registration challenges, including brain shift and visual differences between preoperative MRI-derived renderings and intraoperative images?
    
    \item How does the optimization trajectory differ between intensity-based losses (L1, L2) and structural/information-based losses (SSIM, NCC, MI)?
\end{enumerate}

This research is guided by the hypothesis that more sophisticated loss functions—particularly those capturing structural similarities rather than pixel-wise differences—will demonstrate faster convergence and greater resilience to domain gaps between rendered and intraoperative images. 

This study is complemented by our novel implementation contribution: a model-agnostic neural registration framework built on top of \textbf{nerfstudio} \parencite{Tancik_2023} that enables direct comparison of different loss functions without requiring retraining of the underlying NeRF representations.

\section{Thesis Structure}\label{section:thesis-structure}

The remainder of this thesis is organized as follows:

\begin{itemize}
    \item \textbf{Chapter~\ref{chapter:background}}: Provides the necessary background on neural radiance fields, pose estimation techniques, and current approaches to intraoperative registration. This chapter also reviews relevant literature on loss functions in image registration tasks.
    
    \item \textbf{Chapter~\ref{chapter:methodology}}: Explains the methodology implemented in this work, including the neural radiance field architecture, the formulation of different loss functions, and the optimization framework for registration.
        
    \item \textbf{Chapter~\ref{chapter:results}}: Presents quantitative and qualitative results of our experiments, analyzing the performance of each loss function across multiple metrics and scenarios.
    
    \item \textbf{Chapter~\ref{chap:discussion}}: Discusses the implications of our findings for intraoperative registration, interprets the comparative advantages of each loss function, and considers the practical implications for clinical adoption.
    
    \item \textbf{Chapter~\ref{chapter:conclusion}}: Summarizes the key contributions and insights of this research, acknowledges limitations, and suggests promising directions for future work in neural rendering-based surgical navigation.
\end{itemize}
