% !TeX root = ../main.tex

\chapter{Style Transfer for Cross-Modal Registration}\label{chapter:style_transfer}

This chapter explores the use of style transfer techniques to enhance cross-modal registration between preoperative MRI data and intraoperative optical images. We examine how adapting the appearance of NeRF-rendered images through style transfer can bridge the modality gap and improve registration accuracy. Various style encoding methods and hypernetwork architectures are investigated to determine their effectiveness in the context of neurosurgical registration.

\section{The Cross-Modal Appearance Gap}

One of the fundamental challenges in intraoperative registration is the significant difference in appearance between preoperative MRI data and intraoperative optical images:

\begin{itemize}
    \item \textbf{MRI data} typically presents in grayscale with contrast determined by tissue properties such as T1/T2 relaxation times and proton density.
    
    \item \textbf{Intraoperative optical images} capture the visible surface of the brain with natural color, specular highlights, shadows, and various lighting effects.
\end{itemize}

This appearance gap complicates the registration process, as conventional similarity metrics may struggle to establish correspondence between such visually different representations. Style transfer techniques offer a promising approach to address this challenge by transforming the appearance of NeRF-rendered images to match the visual characteristics of intraoperative images while preserving the underlying anatomical structure.

\section{Hypernetwork-Based Style Control}

Building on the approach introduced by \textcite{fehrentz2024intraoperative}, we employ hypernetworks as a mechanism for controlling the appearance of NeRF renderings. A hypernetwork is a neural network that generates parameters for another neural network (in this case, the NeRF model) based on conditioning information (in this case, style descriptors).

\subsection{Architecture Overview}

Our hypernetwork-based style control system consists of three main components:

\begin{enumerate}
    \item \textbf{Style Encoder}: Extracts style information from reference images or generates style codes from abstract style specifications.
    
    \item \textbf{Hypernetwork}: Generates parameters for a subset of the NeRF model based on the encoded style.
    
    \item \textbf{Parameter Integration}: Incorporates the generated parameters into the NeRF model to influence its rendering appearance.
\end{enumerate}

Figure~\ref{fig:hypernetwork} illustrates this architecture.

% Placeholder for hypernetwork figure
\begin{figure}[htpb]
  \centering
  % Add actual figure in the implementation
  \caption{Architecture of the hypernetwork-based style control system. The style encoder extracts style information from reference images, which the hypernetwork transforms into parameters for the NeRF model, controlling the appearance of rendered images.}
  \label{fig:hypernetwork}
\end{figure}

\subsection{Hypernetwork Design}

We explore several design choices for the hypernetwork:

\begin{itemize}
    \item \textbf{Network Depth and Width}: We compare hypernetworks of different sizes to balance expressiveness and computational efficiency.
    
    \item \textbf{Parameter Generation Scope}: We investigate which NeRF model parameters should be generated by the hypernetwork to effectively control appearance while preserving structure.
    
    \item \textbf{Conditioning Mechanisms}: We experiment with different ways of incorporating style information, including feature concatenation, adaptive instance normalization (AdaIN), and modulation-based approaches.
\end{itemize}

Our findings indicate that generating parameters for the color prediction layers while keeping the density prediction layers fixed provides the best balance between appearance control and structural preservation.

\section{Style Encoding Methods}

A critical aspect of our approach is how style information is encoded and represented. We explore various methods for extracting and encoding style information from reference images:

\subsection{Y'UV Color Space Encoding}

The Y'UV color space separates luminance (Y') from chrominance (U, V), allowing for more intuitive control over color characteristics. Our Y'UV encoding approach includes:

\begin{itemize}
    \item \textbf{Color Space Transformation}: Converting RGB images to Y'UV space.
    
    \item \textbf{Statistical Moments}: Capturing the mean and variance of each channel across the image.
    
    \item \textbf{Histogram Matching}: Aligning the distributions of Y'UV channels between rendered and target images.
\end{itemize}

This method is particularly effective for handling global color differences while requiring relatively low computational resources.

\subsection{Histogram of Oriented Gradients (HOG)}

HOG features capture the distribution of gradient orientations in local regions of an image, providing information about edge patterns and texture directions. Our HOG-based style encoding includes:

\begin{itemize}
    \item \textbf{Feature Extraction}: Computing HOG features at multiple scales.
    
    \item \textbf{Feature Aggregation}: Pooling HOG descriptors to create a compact style representation.
    
    \item \textbf{Feature Integration}: Incorporating HOG information into the style code.
\end{itemize}

HOG encoding helps preserve edge structures and directional patterns, which are important for registration accuracy, especially in regions with distinct vascular patterns.

\subsection{Texture Features and Gabor Filters}

Texture is a key characteristic that differs between MRI and optical images. We explore texture-based style encoding using:

\begin{itemize}
    \item \textbf{Gabor Filter Banks}: Applying filters at multiple orientations and scales to capture texture patterns.
    
    \item \textbf{Local Binary Patterns (LBP)}: Extracting rotation-invariant texture descriptors.
    
    \item \textbf{Texture Moments}: Computing statistical moments of filter responses to create compact texture representations.
\end{itemize}

These texture features help model the fine-grained appearance differences between imaging modalities, particularly for tissue textures that are not captured by simpler color-based encodings.

\subsection{Edge Detection and Contour Matching}

Edges and contours often represent anatomically meaningful boundaries that are consistent across modalities. Our edge-based encoding includes:

\begin{itemize}
    \item \textbf{Edge Detection}: Applying operators such as Canny, Sobel, or learned edge detectors.
    
    \item \textbf{Contour Extraction}: Identifying continuous contours in both rendered and target images.
    
    \item \textbf{Contour Matching}: Aligning contour representations across modalities.
\end{itemize}

This approach helps preserve anatomical boundaries and structural information during style transfer, which is crucial for accurate registration.

\subsection{Gram Matrix for Style Representation}

Inspired by neural style transfer methods, we investigate the use of Gram matrices to capture style information:

\begin{itemize}
    \item \textbf{Feature Extraction}: Using a pre-trained convolutional neural network to extract features from images.
    
    \item \textbf{Gram Matrix Computation}: Calculating the correlation between features at different channels to create a style representation.
    
    \item \textbf{Multi-layer Representation}: Computing Gram matrices at multiple layers to capture both fine and coarse stylistic elements.
\end{itemize}

Gram matrices effectively capture texture patterns and style information at different scales, providing a rich representation for style transfer.

\subsection{Deep Feature Matching}

We explore the use of features extracted from pre-trained convolutional neural networks (CNNs) for style encoding:

\begin{itemize}
    \item \textbf{Feature Extraction}: Using networks pre-trained on medical imaging or natural image datasets.
    
    \item \textbf{Feature Selection}: Identifying the most relevant features for cross-modal matching.
    
    \item \textbf{Feature Adaptation}: Fine-tuning the feature extraction process for the specific task of neurosurgical image matching.
\end{itemize}

Deep features can capture high-level semantic information that may be consistent across modalities, potentially improving the registration of anatomical structures.

\section{Structural Similarity Preservation}

While adapting appearance, it is crucial to preserve the structural information that is essential for accurate registration. We implement several mechanisms to ensure structural similarity:

\subsection{Structural Similarity Index (SSIM)}

We incorporate the Structural Similarity Index as both an evaluation metric and a loss component to preserve structural information:

\begin{equation}
    \text{SSIM}(x, y) = \frac{(2\mu_x\mu_y + C_1)(2\sigma_{xy} + C_2)}{(\mu_x^2 + \mu_y^2 + C_1)(\sigma_x^2 + \sigma_y^2 + C_2)}
\end{equation}

where $\mu_x$, $\mu_y$ are the means, $\sigma_x$, $\sigma_y$ are the standard deviations, and $\sigma_{xy}$ is the covariance of patches from images $x$ and $y$. The constants $C_1$ and $C_2$ prevent division by zero.

\subsection{Structure-Preserving Constraints}

We implement additional constraints to ensure that style transfer does not alter structural information:

\begin{itemize}
    \item \textbf{Density Preservation}: Keeping the density predictions of the NeRF model fixed while only modifying color predictions.
    
    \item \textbf{Edge Alignment}: Ensuring that edges in the rendered images align with those in the target images, regardless of appearance differences.
    
    \item \textbf{Gradient Consistency}: Maintaining consistent gradient directions between original and style-transferred renderings.
\end{itemize}

\section{Style Optimization during Registration}

During the registration process, we need to simultaneously optimize camera pose and style parameters. We explore different strategies for this joint optimization:

\subsection{Sequential Optimization}

The sequential approach alternates between optimizing style parameters and camera pose:

\begin{enumerate}
    \item \textbf{Style Adaptation Phase}: Fix camera pose and optimize style parameters for a few iterations.
    
    \item \textbf{Pose Optimization Phase}: Fix style parameters and optimize camera pose for a few iterations.
    
    \item \textbf{Repeat}: Continue alternating until convergence.
\end{enumerate}

This approach can help avoid local minima by separating the optimization of appearance and geometry.

\subsection{Joint Optimization}

The joint approach optimizes both style parameters and camera pose simultaneously:

\begin{equation}
    \hat{\xi}, \hat{s} = \arg\min_{\xi, s} \mathcal{L}(I_{\text{target}}, I_{\text{rendered}}(\xi, s))
\end{equation}

This approach is more efficient but requires careful balancing of gradients to prevent one set of parameters from dominating the optimization.

\subsection{Multi-Stage Optimization}

The multi-stage approach breaks the optimization into distinct phases:

\begin{enumerate}
    \item \textbf{Initial Style Adaptation}: Optimize style parameters with a fixed initial pose.
    
    \item \textbf{Coarse Pose Optimization}: Optimize camera pose at a low resolution with fixed style.
    
    \item \textbf{Refinement}: Jointly fine-tune both pose and style at higher resolution.
\end{enumerate}

This approach combines the benefits of sequential and joint optimization while addressing their limitations.

\section{Experimental Evaluation of Style Transfer Methods}

To evaluate the effectiveness of different style encoding methods and optimization strategies, we conduct a series of experiments:

\subsection{Style Transfer Quality Assessment}

We assess the quality of style transfer using both quantitative metrics and qualitative evaluation:

\begin{itemize}
    \item \textbf{Appearance Similarity}: Measures how well the rendered images match the target style.
    
    \item \textbf{Structure Preservation}: Evaluates whether structural information is preserved during style transfer.
    
    \item \textbf{Artifact Analysis}: Examines the presence of artifacts or distortions introduced by style transfer.
\end{itemize}

\subsection{Registration Performance with Style Transfer}

We evaluate how different style transfer methods affect registration performance:

\begin{itemize}
    \item \textbf{Registration Accuracy}: Comparing pose estimation accuracy with and without style transfer.
    
    \item \textbf{Convergence Behavior}: Analyzing how style transfer affects optimization convergence.
    
    \item \textbf{Robustness to Initial Conditions}: Testing registration performance with various initial poses.
\end{itemize}

\subsection{Comparison of Style Encoding Methods}

We compare the performance of different style encoding methods in the context of intraoperative registration:

\begin{itemize}
    \item \textbf{Y'UV vs. HOG vs. Texture Features}: Evaluating which encoding methods work best for different types of images.
    
    \item \textbf{Simple vs. Complex Encodings}: Analyzing the trade-off between computational complexity and performance.
    
    \item \textbf{Single vs. Combined Encodings}: Testing whether combining multiple encoding methods yields better results.
\end{itemize}

\section{Results and Analysis}

The detailed results of our style transfer experiments are presented in Chapter~\ref{chapter:results}. However, preliminary findings indicate that:

\begin{itemize}
    \item Y'UV color space encoding provides effective global color adaptation while being computationally efficient.
    
    \item HOG and texture features improve registration in regions with distinctive vessel patterns.
    
    \item Gram matrices capture rich stylistic information but require more computational resources.
    
    \item Multi-stage optimization strategy generally outperforms both sequential and joint optimization approaches.
    
    \item Style transfer significantly improves registration accuracy in cases with substantial appearance differences between preoperative and intraoperative images.
\end{itemize}

\section{Summary}

This chapter has presented a comprehensive exploration of style transfer techniques for enhancing cross-modal registration using Neural Radiance Fields. We have detailed various style encoding methods, from simple color space transformations to complex deep feature representations, and analyzed their effectiveness in bridging the appearance gap between preoperative MRI data and intraoperative optical images.

The integration of style transfer with NeRF-based registration through hypernetworks provides a powerful framework for adapting appearance while preserving structural information. Our experimental results suggest that this approach can significantly improve registration accuracy in the challenging context of neurosurgical navigation.

The next chapter will describe the experimental setup used to evaluate both the loss functions discussed in Chapter~\ref{chapter:loss_functions} and the style transfer techniques presented in this chapter. 