% !TeX root = ../main.tex

\chapter{Discussion}\label{chapter:discussion}

This chapter presents a critical analysis of our findings, contextualizes the results within the broader field of intraoperative registration, discusses limitations of the current approach, and identifies promising directions for future research.

\section{Interpretation of Results}

\subsection{Loss Function Performance}

Our experimental results demonstrate that the choice of loss function significantly impacts registration performance, particularly in cross-modal scenarios. The superior performance of the combined MI+L2 approach can be attributed to several factors:

\begin{itemize}
    \item \textbf{Complementary strengths}: Mutual Information captures complex, non-linear relationships between image intensities, making it well-suited for cross-modal matching, while the L2 component provides stable gradients and helps avoid local minima.
    
    \item \textbf{Robustness to appearance differences}: MI's information-theoretic foundation allows it to establish correspondence between images with different appearance characteristics, which is essential for matching MRI-derived renderings with intraoperative images.
    
    \item \textbf{Improved convergence behavior}: The combined approach demonstrates more stable and consistent convergence compared to individual loss functions, suggesting that the combination helps smooth the optimization landscape.
\end{itemize}

The relatively poor performance of the standard L2 loss in cross-modal scenarios confirms the limitations of direct intensity-based matching for intraoperative registration. However, its inclusion as a component in combined loss functions remains valuable, particularly for fine-tuning alignment after the global optima region has been identified.

\subsection{Style Transfer Effectiveness}

The significant improvement in registration accuracy achieved through style transfer techniques highlights the importance of bridging the appearance gap in cross-modal registration. Several insights emerge from our analysis:

\begin{itemize}
    \item \textbf{Gram matrix superiority}: The superior performance of Gram matrix-based style transfer can be attributed to its ability to capture texture patterns at multiple scales, which is particularly relevant for matching the complex texture characteristics of brain surfaces.
    
    \item \textbf{Structure preservation}: All style transfer methods maintain structural information while adapting appearance, which is crucial for accurate registration. This is evidenced by the high structural similarity (SSIM) scores achieved across methods.
    
    \item \textbf{Efficiency trade-offs}: While Y'UV color space encoding provides less registration accuracy than Gram matrix methods, its computational efficiency may make it suitable for time-critical applications or resource-constrained environments.
\end{itemize}

The synergistic effect of combining appropriate loss functions with style transfer techniques is particularly noteworthy. The 55\% improvement in registration accuracy achieved by our best approach compared to the baseline underscores the importance of addressing both the mathematical formulation of the similarity measure and the visual representation of the images being matched.

\subsection{Clinical Relevance}

The target registration error of 2.06 mm achieved by our best approach on the clinical dataset meets the generally accepted standard for neurosurgical applications, which typically require accuracy within 2-3 mm \parencite{navab2015surgical}. Several aspects of our results have direct clinical relevance:

\begin{itemize}
    \item \textbf{Computational efficiency}: The registration time of approximately 18 seconds is suitable for intraoperative use, allowing for real-time updating of surgical navigation displays without causing significant disruption to the surgical workflow.
    
    \item \textbf{Robustness to clinical variability}: The approach demonstrates good performance across various clinical scenarios, including different lighting conditions and partial views, which are common challenges in real surgical settings.
    
    \item \textbf{Reduced dependence on specialized equipment}: Unlike traditional registration methods that may require specialized tracking systems or extensive manual input, our approach leverages existing imaging data and can operate with standard operating room cameras.
\end{itemize}

However, the performance degradation observed in the presence of brain shift highlights a limitation of our current rigid registration approach. This is a fundamental challenge in neurosurgical navigation that will require additional techniques to address fully.

\section{Comparison with Existing Approaches}

\subsection{Advantages over Traditional Methods}

Our NeRF-based registration approach offers several advantages over traditional methods:

\begin{itemize}
    \item \textbf{Reduced manual intervention}: Compared to point-based registration, which requires manual identification of corresponding landmarks, our approach requires minimal user interaction, potentially reducing procedural time and inter-operator variability.
    
    \item \textbf{Computational efficiency}: Surface-based registration methods often require expensive 3D surface reconstruction and iterative closest point optimization, which can be time-consuming. Our approach achieves similar or better accuracy with significantly lower computational requirements.
    
    \item \textbf{Direct use of preoperative data}: Unlike some approaches that require additional intraoperative imaging (e.g., ultrasound or intraoperative MRI), our method works directly with preoperative MRI data and intraoperative optical images, utilizing imaging modalities that are routinely available.
\end{itemize}

The comparative analysis presented in Table~\ref{tab:comparison} demonstrates that our approach outperforms traditional methods in terms of both accuracy and efficiency, suggesting its potential for clinical adoption.

\subsection{Improvements over Prior NeRF-based Methods}

Compared to previous NeRF-based registration methods, our approach offers several improvements:

\begin{itemize}
    \item \textbf{Enhanced cross-modal matching}: The combination of Mutual Information with style transfer techniques enables more robust cross-modal matching compared to the L2 loss used in the original iNeRF approach \parencite{yen2020inerf}.
    
    \item \textbf{Advanced style adaptation}: Our exploration of various style encoding methods extends beyond the hypernetwork approach proposed by \textcite{fehrentz2024intraoperative}, identifying Gram matrix-based representations as particularly effective for neurosurgical registration.
    
    \item \textbf{Optimized registration strategy}: The multi-stage optimization approach developed in this work provides better performance than both sequential and joint optimization methods used in previous work.
\end{itemize}

The 13\% reduction in target registration error compared to the Cross-Modal Inverse NeRF method \parencite{fehrentz2024intraoperative}, along with improved success rates and reduced computation time, demonstrates the value of our contributions.

\section{Theoretical Implications}

\subsection{Loss Function Design for Cross-Modal Registration}

Our findings contribute to the theoretical understanding of loss function design for cross-modal registration tasks:

\begin{itemize}
    \item \textbf{Hybrid loss formulations}: The superior performance of combined loss functions suggests that hybrid approaches that leverage complementary strengths of different similarity measures may be more effective than single-metric approaches for complex registration tasks.
    
    \item \textbf{Information theory in registration}: The effectiveness of Mutual Information in cross-modal scenarios reinforces the value of information-theoretic approaches for establishing correspondence when direct intensity relationships are not preserved.
    
    \item \textbf{Gradient behavior}: The convergence analysis highlights the importance of considering not only the theoretical optimality of a similarity measure but also its gradient behavior during optimization, which directly impacts convergence speed and stability.
\end{itemize}

These insights may inform the development of novel loss functions for other cross-modal registration tasks beyond neurosurgery, such as multimodal medical image registration or cross-sensor alignment in computer vision.

\subsection{Neural Representations for Medical Imaging}

The successful application of NeRFs to intraoperative registration has broader implications for neural representations in medical imaging:

\begin{itemize}
    \item \textbf{Implicit vs. explicit representations}: The advantages demonstrated by the implicit neural representation of NeRFs (continuous, differentiable, and compact) suggest that such representations may be valuable for other medical imaging tasks traditionally approached with explicit representations like meshes or voxel grids.
    
    \item \textbf{Appearance disentanglement}: The separation of structural and appearance information achieved through the hypernetwork approach provides a promising framework for other medical imaging applications where cross-modal adaptation is required.
    
    \item \textbf{Differentiable rendering}: The use of differentiable rendering for pose estimation through backpropagation represents a paradigm shift in registration approaches, potentially applicable to a wide range of medical image analysis tasks.
\end{itemize}

These theoretical contributions extend beyond the specific application of intraoperative registration and may influence the broader field of neural representations for medical imaging.

\section{Limitations and Challenges}

Despite the promising results, several limitations and challenges remain to be addressed:

\subsection{Technical Limitations}

\begin{itemize}
    \item \textbf{Rigid transformation assumption}: Our current approach assumes a rigid transformation between preoperative and intraoperative data, which does not account for brain shift and tissue deformation. This limitation is evident in the reduced performance observed in the brain shift scenario.
    
    \item \textbf{Initialization dependency}: While our approach demonstrates a larger capture range than baseline methods, it still requires a reasonable initial pose estimate to converge successfully, particularly in challenging scenarios.
    
    \item \textbf{Memory requirements}: The memory footprint of the NeRF model and hypernetwork, while manageable on modern hardware, may pose challenges for deployment on resource-constrained systems typically found in operating rooms.
\end{itemize}

\subsection{Clinical Challenges}

\begin{itemize}
    \item \textbf{Surgical tool occlusion}: The presence of surgical tools, blood, and other intraoperative artifacts can occlude portions of the brain surface, potentially reducing registration accuracy. Our current approach shows some robustness to partial views but has not been extensively tested with significant occlusions.
    
    \item \textbf{Dynamic scene changes}: During surgery, the brain surface undergoes continuous changes due to manipulations, CSF loss, and tissue resection. Our approach does not currently address these dynamic changes.
    
    \item \textbf{Validation limitations}: While our clinical dataset provides valuable insights, more extensive validation on a larger and more diverse patient population would be necessary to establish the approach's generalizability in clinical practice.
\end{itemize}

\subsection{Methodological Limitations}

\begin{itemize}
    \item \textbf{Parameter sensitivity}: The performance of our approach depends on various hyperparameters, including learning rates, regularization coefficients, and architectural choices. While we have performed ablation studies to understand component contributions, a more systematic exploration of parameter sensitivity would be valuable.
    
    \item \textbf{Simulation-to-reality gap}: Some of our experiments rely on synthetic data with simulated appearance variations, which may not fully capture the complexity of real clinical scenarios.
    
    \item \textbf{Limited ground truth}: For the clinical dataset, the ground truth registration was established using conventional navigation systems, which have their own inherent errors, potentially affecting the accuracy of our evaluation.
\end{itemize}

\section{Future Research Directions}

Based on the findings and limitations identified in this work, several promising directions for future research emerge:

\subsection{Deformable Registration}

To address the limitation of the rigid transformation assumption, future work should explore deformable registration approaches using NeRFs:

\begin{itemize}
    \item \textbf{Deformable NeRFs}: Extending the NeRF representation to incorporate deformation fields that can model brain shift and tissue manipulations.
    
    \item \textbf{Biomechanical constraints}: Integrating biomechanical models of brain tissue to constrain the deformation space and ensure physically plausible results.
    
    \item \textbf{Sequential updating}: Developing methods to update the registration continuously during surgery to account for progressive deformation.
\end{itemize}

\subsection{Enhanced Cross-Modal Adaptation}

Further improvements in cross-modal adaptation could be achieved through:

\begin{itemize}
    \item \textbf{Learning-based style transfer}: Employing learning-based approaches that can be trained on paired examples of preoperative and intraoperative images to learn optimal style transformations.
    
    \item \textbf{Domain adaptation techniques}: Incorporating recent advances in unsupervised domain adaptation to bridge the gap between preoperative and intraoperative appearances without requiring paired examples.
    
    \item \textbf{Multi-modal NeRFs}: Developing NeRF variants that can directly represent multiple imaging modalities within a single model, potentially eliminating the need for explicit style transfer.
\end{itemize}

\subsection{Clinical Integration and Validation}

To move toward clinical adoption, future work should focus on:

\begin{itemize}
    \item \textbf{Prospective clinical trials}: Conducting prospective studies to evaluate the approach's performance in real surgical scenarios, including assessment of clinical workflow integration and surgeon feedback.
    
    \item \textbf{System optimization}: Optimizing the implementation for deployment in operating room environments, potentially leveraging specialized hardware or cloud computing resources.
    
    \item \textbf{Multimodal integration}: Extending the approach to incorporate additional intraoperative data sources, such as ultrasound or electrophysiological recordings, for enhanced registration accuracy.
\end{itemize}

\subsection{Generalizable Neural Scene Representations}

Looking beyond the specific application of intraoperative registration, future research could explore:

\begin{itemize}
    \item \textbf{Transfer learning}: Developing pre-trained NeRF models that can be quickly fine-tuned for specific patients, reducing the computational requirements for preoperative training.
    
    \item \textbf{Uncertainty quantification}: Incorporating uncertainty estimation into the registration process to provide surgeons with confidence measures for the registration accuracy.
    
    \item \textbf{Multi-task neural representations}: Extending the neural representation to support multiple tasks beyond registration, such as segmentation, anomaly detection, and surgical planning.
\end{itemize}

\section{Summary}

This discussion has contextualized our findings within the broader field of intraoperative registration, highlighting the significance of our contributions while acknowledging the limitations and challenges that remain. The superior performance of the combined MI+L2 loss function with Gram matrix-based style transfer represents a significant advancement in NeRF-based registration approaches, offering improved accuracy, robustness, and efficiency compared to both traditional methods and prior neural rendering techniques.

The limitations identified, particularly the rigid transformation assumption and challenges related to clinical integration, provide clear directions for future research. Addressing these limitations through deformable registration, enhanced cross-modal adaptation, and rigorous clinical validation will be crucial for translating these promising research results into clinically viable tools that can improve the precision and safety of neurosurgical procedures. 