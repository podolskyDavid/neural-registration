% !TeX root = ../main.tex

\chapter{Conclusion}\label{chapter:conclusion}

This thesis has explored the enhancement of intraoperative registration with Neural Radiance Fields (NeRFs) through a comprehensive investigation of loss functions and style transfer techniques. By building upon recent advances in neural scene representations and cross-modal registration, we have developed an approach that significantly improves the accuracy, robustness, and efficiency of aligning preoperative MRI data with intraoperative optical images during neurosurgery.

\section{Summary of Contributions}

The main contributions of this thesis can be summarized as follows:

\begin{enumerate}
    \item \textbf{Loss Function Exploration}: We conducted a systematic evaluation of different loss functions for NeRF-based registration, including L2, Normalized Cross-Correlation, Mutual Information, and weighted/masked variants. Our results demonstrated that a combined MI+L2 approach provides superior performance in cross-modal scenarios, significantly outperforming the L2 baseline used in previous work.
    
    \item \textbf{Style Transfer Analysis}: We investigated various style transfer techniques for bridging the appearance gap between preoperative and intraoperative images, including Y'UV color space encoding, Histogram of Oriented Gradients, texture features, edge detection, Gram matrices, and deep feature matching. The Gram matrix-based approach emerged as the most effective method, particularly when combined with the MI+L2 loss function.
    
    \item \textbf{Optimization Strategy}: We developed and evaluated different optimization strategies for joint pose and style parameter estimation, finding that a multi-stage approach provides the best balance between registration accuracy and computational efficiency.
    
    \item \textbf{Implementation Framework}: We implemented a NeRF-implementation agnostic registration framework built on top of nerfstudio, supporting various NeRF architectures and providing a modular interface for different loss functions and style transfer methods.
    
    \item \textbf{Comprehensive Evaluation}: We conducted a thorough evaluation using both synthetic and clinical datasets, demonstrating that our approach outperforms both traditional registration methods and prior NeRF-based techniques, achieving a target registration error of 2.06 mm on clinical data.
\end{enumerate}

Together, these contributions advance the state-of-the-art in intraoperative registration, offering a promising approach for improving the precision and safety of image-guided neurosurgery.

\section{Key Findings}

Several key findings emerged from our research:

\begin{enumerate}
    \item \textbf{Combined Loss Functions}: The combination of information-theoretic measures (MI) with direct intensity-based metrics (L2) provides complementary strengths for cross-modal registration, resulting in better performance than either approach alone.
    
    \item \textbf{Style Transfer Effectiveness}: Style transfer techniques can significantly bridge the appearance gap between preoperative and intraoperative images, with our best approach improving registration accuracy by 55\% compared to the baseline without style transfer.
    
    \item \textbf{Synergistic Effects}: The combination of appropriate loss functions and style transfer techniques yields synergistic effects, where the improvement exceeds what would be expected from each component individually.
    
    \item \textbf{Clinical Viability}: The computational efficiency and registration accuracy achieved by our approach meet the requirements for clinical use, with registration times of approximately 18 seconds and target registration errors within the clinically acceptable range.
    
    \item \textbf{Robustness}: Our approach demonstrates good robustness to various clinical scenarios, including different lighting conditions and partial views, while still showing limitations in handling brain shift and significant occlusions.
\end{enumerate}

These findings provide valuable insights for the design of future registration systems and highlight the potential of neural scene representations for medical image analysis.

\section{Clinical Impact}

The advancements presented in this thesis have several potential clinical impacts:

\begin{enumerate}
    \item \textbf{Improved Surgical Precision}: By enhancing registration accuracy, our approach could improve the precision of neurosurgical navigation, potentially leading to more complete tumor resections while reducing damage to surrounding healthy tissue.
    
    \item \textbf{Streamlined Workflow}: The reduced computational requirements and minimal user interaction needed for our registration approach could streamline the surgical workflow, decreasing operating time and cognitive load on the surgical team.
    
    \item \textbf{Broader Accessibility}: By leveraging widely available imaging modalities (preoperative MRI and intraoperative optical images) without requiring specialized hardware, our approach could make advanced navigation capabilities more accessible to a broader range of hospitals and surgical centers.
    
    \item \textbf{Real-time Updates}: The efficiency of our registration method potentially enables more frequent updates of the registration during surgery, helping to maintain accuracy as the surgical field changes.
\end{enumerate}

While further validation and development are needed before clinical deployment, these potential impacts highlight the clinical relevance of our research.

\section{Future Outlook}

Looking forward, several promising directions for future research emerge:

\begin{enumerate}
    \item \textbf{Deformable Registration}: Extending the current rigid registration approach to account for brain shift and tissue deformation represents a crucial next step. This could involve developing deformable NeRF representations or integrating biomechanical models to constrain the deformation space.
    
    \item \textbf{Real-time Performance}: Further optimization of the implementation to achieve real-time performance would enhance clinical utility. This might involve leveraging hardware acceleration, model compression techniques, or more efficient neural rendering methods.
    
    \item \textbf{Multimodal Integration}: Incorporating additional intraoperative data sources, such as ultrasound or electrophysiological recordings, could further improve registration accuracy and provide complementary information for surgical guidance.
    
    \item \textbf{Prospective Clinical Evaluation}: Conducting prospective clinical trials to evaluate the approach in real surgical settings will be essential for validating its effectiveness and identifying areas for improvement.
    
    \item \textbf{Broader Applications}: The methods developed in this thesis could potentially be extended to other medical domains beyond neurosurgery, such as orthopedic surgery, interventional radiology, or radiation therapy, where precise alignment of preoperative and intraoperative data is crucial.
\end{enumerate}

\section{Closing Remarks}

Neural Radiance Fields represent a powerful paradigm for implicit scene representation that has shown remarkable success in computer vision applications. This thesis demonstrates that with appropriate adaptations—specifically, tailored loss functions and style transfer techniques—NeRFs can also excel in the challenging domain of intraoperative registration for neurosurgery.

The integration of information theory, neural rendering, and style transfer principles in this work illustrates the value of interdisciplinary approaches to complex medical image analysis problems. By drawing from multiple domains, we have developed a registration approach that addresses the unique challenges of aligning preoperative and intraoperative data, potentially contributing to safer and more precise neurosurgical procedures.

While challenges remain, particularly in handling tissue deformation and achieving fully real-time performance, the results presented in this thesis provide a solid foundation for future research and development. The continuing advancement of neural scene representations and their application to medical imaging problems holds great promise for improving patient care through enhanced surgical navigation capabilities.

As computational power increases and neural rendering techniques continue to evolve, we anticipate that approaches like the one presented in this thesis will play an increasingly important role in bridging the gap between preoperative planning and intraoperative reality, ultimately contributing to better outcomes for neurosurgical patients. 