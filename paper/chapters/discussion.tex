\chapter{Discussion}

This chapter discusses the implications of our experimental results, compares our approach with existing methods, identifies limitations, and suggests directions for future work.

\section{Interpretation of Results}

\subsection{Loss Function Performance}
Our results show that the Structural Similarity Index (SSIM) loss function outperforms other loss functions for NeRF-based intraoperative registration. This can be attributed to SSIM's ability to capture structural information in images, which is particularly important for medical images where anatomical structures need to be aligned precisely. Unlike pixel-wise losses such as L2, SSIM considers luminance, contrast, and structural information, making it more robust to variations in lighting and appearance between the rendered and target images.

The superior performance of SSIM is consistent with findings in other medical image registration tasks, where structural similarity metrics have been shown to be more effective than intensity-based metrics \cite{wang2004image}. However, our results also show that Mutual Information and Normalized Cross-Correlation perform well, suggesting that these metrics could be viable alternatives in scenarios where SSIM computation is too expensive or when dealing with specific types of images.

\subsection{Hypernetwork Effectiveness}
The significant improvement in registration accuracy achieved by hypernetwork-based style transfer approaches demonstrates the importance of addressing the cross-modal nature of intraoperative registration. The Deep Feature Matching approach, which leverages features from pre-trained convolutional neural networks, shows the best performance, likely because these features capture high-level semantic information that is invariant to modality-specific appearance variations.

The effectiveness of hypernetworks in bridging the modality gap between preoperative MRI and intraoperative optical images is a key finding of our work. By learning to modulate the NeRF rendering to match the target modality, hypernetworks enable more accurate registration without requiring explicit correspondence matching or feature extraction during the registration process.

\subsection{Robustness and Convergence}
The combination of SSIM loss and Deep Feature Matching hypernetwork shows remarkable robustness to initial pose perturbations, with a 90\% success rate for perturbations of up to 15.6 mm in translation and 22.7 degrees in rotation. This level of robustness is crucial for clinical applications, where the initial pose estimate may be inaccurate due to brain shift, patient movement, or other factors.

The convergence behavior of different loss functions provides insights into their optimization landscapes. The SSIM and Mutual Information loss functions converge more quickly and to lower error values, suggesting that they create smoother optimization landscapes with fewer local minima. This is particularly important for gradient-based optimization methods, which can get stuck in local minima when using loss functions with complex landscapes.

\section{Comparison with Existing Methods}

\subsection{Traditional Registration Methods}
Compared to traditional mesh-based registration methods, our NeRF-based approach offers several advantages:

\begin{itemize}
    \item \textbf{Differentiability}: The differentiable nature of NeRFs allows for gradient-based optimization of camera poses, enabling more efficient and accurate registration.
    
    \item \textbf{Implicit Representation}: NeRFs provide an implicit representation of the brain surface, eliminating the need for explicit correspondence matching or feature extraction.
    
    \item \textbf{Appearance Modeling}: NeRFs model both geometry and appearance, allowing for more realistic rendering and better matching with intraoperative images.
\end{itemize}

Our results show that our approach achieves a mean Target Registration Error of 2.29 mm with the best configuration, which is comparable to or better than state-of-the-art mesh-based methods that typically report errors in the range of 2-5 mm \cite{maier2017surgical}.

\subsection{Other NeRF-based Methods}
Compared to existing NeRF-based registration methods such as iNeRF \cite{yen2021inerf} and Parallel Inversion \cite{wang2021parallel}, our approach offers several improvements:

\begin{itemize}
    \item \textbf{Loss Function Exploration}: We provide a comprehensive evaluation of different loss functions, showing that SSIM outperforms the L2 loss used in previous work.
    
    \item \textbf{Hypernetwork Integration}: Our integration of hypernetwork-based style transfer addresses the cross-modal nature of intraoperative registration, which is not considered in previous NeRF-based methods.
    
    \item \textbf{Implementation Agnosticism}: Our framework is designed to be agnostic to the specific NeRF implementation, allowing for experimentation with different architectures and rendering techniques.
\end{itemize}

Our results show that these improvements lead to better registration accuracy and robustness compared to previous methods. For example, iNeRF reports rotation errors of 3-5 degrees for general scenes, while our approach achieves a mean rotation error of 2.18 degrees for brain registration, which is a more challenging task due to the complex geometry and appearance of brain surfaces.

\section{Limitations and Challenges}

\subsection{Computational Requirements}
Despite the promising results, our approach has some limitations in terms of computational requirements:

\begin{itemize}
    \item \textbf{Memory Usage}: The Deep Feature Matching hypernetwork has a high memory footprint (1,876 MB), which may be a concern for deployment on systems with limited GPU memory.
    
    \item \textbf{Rendering Time}: NeRF rendering is still relatively slow compared to traditional mesh rendering, with each iteration taking 42-89 ms depending on the method. This could be a bottleneck for real-time applications.
    
    \item \textbf{Training Overhead}: Training the NeRF and hypernetwork models requires significant computational resources and time, which may limit the applicability in scenarios with limited resources.
\end{itemize}

Recent advances in NeRF acceleration, such as Instant-NGP \cite{muller2022instant}, could help address some of these limitations by reducing rendering time and memory usage.

\subsection{Generalization to Real Clinical Data}
While our experiments on real data show promising results, there are still challenges in generalizing to the full range of clinical scenarios:

\begin{itemize}
    \item \textbf{Tissue Deformation}: Our current approach assumes that the brain surface geometry is relatively stable between the preoperative and intraoperative stages. However, significant tissue deformation can occur during surgery, which may require non-rigid registration techniques.
    
    \item \textbf{Partial Visibility}: In real surgeries, only a portion of the brain surface is visible, and the visible region may change during the procedure. This partial visibility can make registration more challenging.
    
    \item \textbf{Surgical Instruments}: The presence of surgical instruments, blood, and other artifacts in intraoperative images can interfere with the registration process. Our current approach does not explicitly handle these artifacts.
\end{itemize}

Addressing these challenges will require extensions to our framework, such as incorporating deformation modeling, handling partial visibility, and developing robust methods for artifact detection and removal.

\subsection{Validation and Clinical Integration}
The clinical validation of our approach is still limited:

\begin{itemize}
    \item \textbf{Sample Size}: Our real dataset consists of only 20 cases, which may not be sufficient to fully evaluate the performance across different patient populations and surgical scenarios.
    
    \item \textbf{Ground Truth}: Establishing ground truth for registration accuracy in real clinical data is challenging, as there is no direct way to measure the true alignment between preoperative and intraoperative images.
    
    \item \textbf{Clinical Workflow Integration}: Integrating our approach into the clinical workflow requires addressing practical considerations such as user interface design, real-time feedback, and compatibility with existing surgical navigation systems.
\end{itemize}

More extensive clinical validation and workflow integration studies are needed to fully assess the potential of our approach for real-world clinical use.

\section{Future Work}

\subsection{Technical Improvements}
Several technical improvements could enhance the performance and applicability of our approach:

\begin{itemize}
    \item \textbf{Real-time Rendering}: Investigating techniques to accelerate NeRF rendering, such as neural caching, model distillation, or hardware-specific optimizations, to enable real-time registration updates during surgery.
    
    \item \textbf{Adaptive Loss Functions}: Developing adaptive loss functions that combine the strengths of different metrics and adjust their weights based on the registration progress and image characteristics.
    
    \item \textbf{Multi-resolution Approach}: Implementing a multi-resolution registration strategy that starts with low-resolution images for global alignment and progressively refines the registration with higher-resolution images.
    
    \item \textbf{Uncertainty Estimation}: Incorporating uncertainty estimation in the registration process to provide confidence measures for the estimated poses, which could be valuable for clinical decision-making.
\end{itemize}

\subsection{Extensions to Handle Tissue Deformation}
To address the challenge of tissue deformation, future work could explore:

\begin{itemize}
    \item \textbf{Deformable NeRFs}: Extending the NeRF representation to model deformable objects, allowing for non-rigid registration of brain surfaces.
    
    \item \textbf{Biomechanical Constraints}: Incorporating biomechanical constraints into the registration process to ensure physically plausible deformations.
    
    \item \textbf{Incremental Registration}: Developing methods for incremental registration that update the NeRF model during surgery to account for progressive deformation.
\end{itemize}

\subsection{Clinical Translation}
To facilitate clinical translation of our approach, future work should focus on:

\begin{itemize}
    \item \textbf{Prospective Clinical Studies}: Conducting prospective clinical studies to evaluate the impact of our registration approach on surgical outcomes, workflow efficiency, and surgeon satisfaction.
    
    \item \textbf{Integration with Surgical Navigation}: Developing interfaces and protocols for integrating our approach with existing surgical navigation systems.
    
    \item \textbf{User-friendly Tools}: Creating user-friendly tools for surgeons to interact with the registration system, provide feedback, and make adjustments as needed during surgery.
    
    \item \textbf{Regulatory Considerations}: Addressing regulatory requirements for medical device approval, including safety, efficacy, and quality assurance aspects.
\end{itemize}

\subsection{Broader Applications}
The techniques developed in this work could be extended to other medical imaging applications:

\begin{itemize}
    \item \textbf{Other Surgical Domains}: Applying our approach to other surgical domains such as orthopedic, abdominal, or cardiac surgery, where registration between preoperative and intraoperative images is also important.
    
    \item \textbf{Longitudinal Registration}: Using NeRF-based registration for tracking changes in anatomical structures over time, which could be valuable for monitoring disease progression or treatment response.
    
    \item \textbf{Multi-modal Fusion}: Extending our hypernetwork approach to facilitate fusion of multiple imaging modalities (e.g., MRI, CT, ultrasound, optical) for comprehensive surgical planning and guidance.
\end{itemize}

\section{Conclusion}
Our work demonstrates the potential of NeRF-based approaches for intraoperative registration, with significant improvements over traditional methods in terms of accuracy, robustness, and cross-modal capability. The combination of SSIM loss and Deep Feature Matching hypernetwork provides the best performance, achieving a mean Target Registration Error of 2.29 mm on real clinical data.

While there are still challenges to overcome, particularly in terms of computational efficiency, tissue deformation handling, and clinical integration, our results suggest that NeRF-based registration could become a valuable tool for image-guided neurosurgery, potentially improving surgical precision and patient outcomes.

Future work should focus on addressing the identified limitations, extending the approach to handle more complex scenarios, and facilitating clinical translation through prospective studies and integration with existing surgical workflows. 