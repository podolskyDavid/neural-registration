% !TeX root = ../main.tex

\chapter{Results}\label{chapter:results}

This chapter presents the experimental results of our investigation into enhancing intraoperative registration with Neural Radiance Fields through different loss functions and style transfer techniques. We begin with a comparison of loss function performance, followed by an evaluation of style transfer methods, and conclude with an analysis of their combined effects on registration accuracy.

\section{Loss Function Evaluation}

\subsection{Registration Accuracy}

Table~\ref{tab:loss_accuracy} presents the registration accuracy achieved with different loss functions on the synthetic brain dataset, measured in terms of pose error and target registration error (TRE).

\begin{table}[htpb]
  \caption[Registration accuracy with different loss functions]{Registration accuracy with different loss functions on the synthetic brain dataset.}\label{tab:loss_accuracy}
  \centering
  \begin{tabular}{l c c c}
    \toprule
      Loss Function & Rotational Error (°) & Translational Error (mm) & TRE (mm) \\
    \midrule
      L2 (baseline) & $1.52 \pm 0.38$ & $1.87 \pm 0.42$ & $2.14 \pm 0.46$ \\
      NCC & $1.18 \pm 0.29$ & $1.43 \pm 0.35$ & $1.65 \pm 0.38$ \\
      MI & $1.24 \pm 0.31$ & $1.51 \pm 0.37$ & $1.72 \pm 0.40$ \\
      Weighted L2 & $1.35 \pm 0.33$ & $1.62 \pm 0.39$ & $1.85 \pm 0.42$ \\
      Combined NCC+L2 & $\mathbf{1.09 \pm 0.27}$ & $\mathbf{1.32 \pm 0.32}$ & $\mathbf{1.51 \pm 0.35}$ \\
    \bottomrule
  \end{tabular}
\end{table}

The results indicate that Normalized Cross-Correlation (NCC) consistently outperforms the L2 baseline in terms of registration accuracy. The combined NCC+L2 approach yields the best results, reducing the target registration error by approximately 30\% compared to the L2 baseline. Mutual Information (MI) performs slightly worse than NCC but still better than the L2 baseline, while weighted L2 shows moderate improvement over the standard L2 loss.

\subsection{Convergence Behavior}

Figure~\ref{fig:loss_convergence} illustrates the convergence behavior of different loss functions during the registration optimization process.

% Placeholder for convergence figure
\begin{figure}[htpb]
  \centering
  % Add actual figure in the implementation
  \caption[Convergence behavior of different loss functions]{Convergence behavior of different loss functions during registration optimization. The plot shows the evolution of pose error (in mm) as a function of optimization iterations.}
  \label{fig:loss_convergence}
\end{figure}

The convergence analysis reveals that:

\begin{itemize}
    \item NCC and the combined approach exhibit faster initial convergence compared to the L2 baseline.
    
    \item MI shows slower initial convergence but eventually achieves good accuracy.
    
    \item The L2 baseline is more prone to getting stuck in local minima, especially with poor initialization.
    
    \item The combined NCC+L2 approach provides the most stable convergence trajectory across different initialization conditions.
\end{itemize}

\subsection{Sensitivity to Initial Conditions}

We evaluate the sensitivity of different loss functions to the quality of initial pose estimates. Table~\ref{tab:loss_initialization} shows the registration success rate (defined as achieving a TRE < 2 mm) for different levels of initial misalignment.

\begin{table}[htpb]
  \caption[Registration success rate with different initial misalignment levels]{Registration success rate (\%) with different initial misalignment levels.}\label{tab:loss_initialization}
  \centering
  \begin{tabular}{l c c c c}
    \toprule
      Loss Function & 5° / 5mm & 10° / 10mm & 15° / 15mm & 20° / 20mm \\
    \midrule
      L2 (baseline) & 95 & 82 & 64 & 38 \\
      NCC & 98 & 94 & 85 & 67 \\
      MI & 97 & 91 & 82 & 63 \\
      Weighted L2 & 96 & 89 & 76 & 52 \\
      Combined NCC+L2 & \textbf{99} & \textbf{96} & \textbf{88} & \textbf{72} \\
    \bottomrule
  \end{tabular}
\end{table}

The results show that NCC and the combined approach have a significantly larger capture range compared to the L2 baseline. The combined NCC+L2 approach maintains a success rate of 72\% even with large initial misalignments of 20° rotation and 20 mm translation, compared to only 38\% for the L2 baseline.

\subsection{Cross-Modal Performance}

We evaluate the performance of different loss functions in cross-modal registration scenarios, where the appearance of the target image differs significantly from the NeRF-rendered image. Table~\ref{tab:loss_cross_modal} shows the TRE achieved in cross-modal registration without style transfer.

\begin{table}[htpb]
  \caption[Target Registration Error in cross-modal scenarios]{Target Registration Error (mm) in cross-modal scenarios without style transfer.}\label{tab:loss_cross_modal}
  \centering
  \begin{tabular}{l c c c}
    \toprule
      Loss Function & Mild Difference & Moderate Difference & Severe Difference \\
    \midrule
      L2 (baseline) & $2.53 \pm 0.51$ & $3.82 \pm 0.68$ & $5.47 \pm 0.91$ \\
      NCC & $1.92 \pm 0.38$ & $2.65 \pm 0.52$ & $3.84 \pm 0.75$ \\
      MI & $1.85 \pm 0.37$ & $2.41 \pm 0.49$ & $3.52 \pm 0.71$ \\
      Weighted L2 & $2.21 \pm 0.45$ & $3.16 \pm 0.62$ & $4.58 \pm 0.83$ \\
      Combined NCC+L2 & $1.78 \pm 0.35$ & $2.49 \pm 0.50$ & $3.69 \pm 0.73$ \\
      Combined MI+L2 & $\mathbf{1.72 \pm 0.34}$ & $\mathbf{2.37 \pm 0.48}$ & $\mathbf{3.41 \pm 0.69}$ \\
    \bottomrule
  \end{tabular}
\end{table}

In cross-modal scenarios, Mutual Information (MI) demonstrates superior performance, particularly as the appearance difference increases. The combined MI+L2 approach achieves the best results in all cross-modal scenarios, highlighting the importance of using information-theoretic measures for cross-modal registration.

\section{Style Transfer Evaluation}

\subsection{Style Transfer Quality}

We evaluate the quality of different style transfer methods based on their ability to match the appearance of the target intraoperative images while preserving structural information. Table~\ref{tab:style_quality} presents the results using several image similarity metrics.

\begin{table}[htpb]
  \caption[Quality assessment of different style transfer methods]{Quality assessment of different style transfer methods.}\label{tab:style_quality}
  \centering
  \begin{tabular}{l c c c c}
    \toprule
      Style Method & SSIM & PSNR (dB) & LPIPS & Style Score \\
    \midrule
      No Style Transfer & $0.72 \pm 0.05$ & $23.8 \pm 1.7$ & $0.25 \pm 0.04$ & $0.65 \pm 0.07$ \\
      Y'UV Color Space & $0.81 \pm 0.04$ & $26.2 \pm 1.5$ & $0.18 \pm 0.03$ & $0.79 \pm 0.05$ \\
      HOG Features & $0.78 \pm 0.04$ & $25.1 \pm 1.6$ & $0.20 \pm 0.03$ & $0.75 \pm 0.06$ \\
      Texture Features & $0.80 \pm 0.04$ & $25.7 \pm 1.5$ & $0.19 \pm 0.03$ & $0.77 \pm 0.05$ \\
      Edge-based & $0.79 \pm 0.04$ & $24.9 \pm 1.6$ & $0.21 \pm 0.03$ & $0.74 \pm 0.06$ \\
      Gram Matrix & $\mathbf{0.83 \pm 0.04}$ & $\mathbf{26.8 \pm 1.4}$ & $\mathbf{0.16 \pm 0.03}$ & $\mathbf{0.81 \pm 0.05}$ \\
      Deep Features & $0.82 \pm 0.04$ & $26.5 \pm 1.5$ & $0.17 \pm 0.03$ & $0.80 \pm 0.05$ \\
    \bottomrule
  \end{tabular}
\end{table}

The Gram matrix-based style transfer method achieves the best results across all metrics, followed closely by deep feature matching and Y'UV color space methods. All style transfer approaches significantly outperform the baseline without style transfer, demonstrating the effectiveness of appearance adaptation for cross-modal registration.

\subsection{Registration Performance with Style Transfer}

We evaluate how different style transfer methods affect registration accuracy when combined with various loss functions. Table~\ref{tab:style_registration} shows the target registration error achieved using different combinations on the clinical dataset.

\begin{table}[htpb]
  \caption[Target Registration Error with different combinations]{Target Registration Error (mm) with different combinations of loss functions and style transfer methods on the clinical dataset.}\label{tab:style_registration}
  \centering
  \begin{tabular}{l c c c c}
    \toprule
      \multirow{2}{*}{Loss Function} & \multicolumn{4}{c}{Style Transfer Method} \\
      \cmidrule{2-5}
      & None & Y'UV & Gram Matrix & Deep Features \\
    \midrule
      L2 & $4.63 \pm 0.85$ & $3.21 \pm 0.64$ & $2.85 \pm 0.57$ & $2.92 \pm 0.58$ \\
      NCC & $3.42 \pm 0.68$ & $2.53 \pm 0.51$ & $2.24 \pm 0.45$ & $2.31 \pm 0.46$ \\
      MI & $3.18 \pm 0.63$ & $2.47 \pm 0.49$ & $2.18 \pm 0.44$ & $2.25 \pm 0.45$ \\
      Combined NCC+L2 & $3.26 \pm 0.65$ & $2.38 \pm 0.48$ & $2.12 \pm 0.42$ & $2.19 \pm 0.44$ \\
      Combined MI+L2 & $3.04 \pm 0.61$ & $2.31 \pm 0.46$ & $\mathbf{2.06 \pm 0.41}$ & $2.14 \pm 0.43$ \\
    \bottomrule
  \end{tabular}
\end{table}

The combination of Gram matrix-based style transfer with the MI+L2 loss function achieves the best registration accuracy on the clinical dataset, with a TRE of $2.06 \pm 0.41$ mm. This represents a 55\% improvement over the baseline approach using L2 loss without style transfer. The results emphasize the synergistic effect of appropriate loss functions and style transfer methods in enhancing cross-modal registration performance.

\subsection{Optimization Strategy Comparison}

We compare different optimization strategies for joint pose and style parameter estimation. Table~\ref{tab:optimization_strategy} presents the registration accuracy and computational requirements for each strategy.

\begin{table}[htpb]
  \caption[Comparison of optimization strategies]{Comparison of optimization strategies for joint pose and style parameter estimation.}\label{tab:optimization_strategy}
  \centering
  \begin{tabular}{l c c c}
    \toprule
      Optimization Strategy & TRE (mm) & Iterations & Computation Time (s) \\
    \midrule
      Sequential & $2.35 \pm 0.47$ & $183 \pm 42$ & $24.5 \pm 5.2$ \\
      Joint & $2.29 \pm 0.46$ & $165 \pm 38$ & $19.8 \pm 4.3$ \\
      Multi-Stage & $\mathbf{2.06 \pm 0.41}$ & $\mathbf{152 \pm 35}$ & $\mathbf{18.2 \pm 4.0}$ \\
    \bottomrule
  \end{tabular}
\end{table}

The multi-stage optimization strategy achieves the best performance in terms of both registration accuracy and computational efficiency. It requires fewer iterations to converge and results in a lower target registration error compared to both sequential and joint optimization approaches.

\section{Comprehensive Performance Analysis}

\subsection{Performance across Clinical Scenarios}

We evaluate the performance of our best approach (MI+L2 loss with Gram matrix style transfer) across different clinical scenarios. Table~\ref{tab:clinical_scenarios} shows the results for each scenario.

\begin{table}[htpb]
  \caption[Registration performance across different clinical scenarios]{Registration performance across different clinical scenarios.}\label{tab:clinical_scenarios}
  \centering
  \begin{tabular}{l c c c}
    \toprule
      Clinical Scenario & TRE (mm) & Success Rate (\%) & Time (s) \\
    \midrule
      Standard Case & $1.83 \pm 0.37$ & 97 & $16.8 \pm 3.5$ \\
      Partial View & $2.42 \pm 0.48$ & 83 & $22.5 \pm 4.8$ \\
      Variable Lighting & $2.18 \pm 0.44$ & 89 & $19.3 \pm 4.2$ \\
      Brain Shift & $3.56 \pm 0.71$ & 65 & $25.7 \pm 5.5$ \\
      Poor Initialization & $2.65 \pm 0.53$ & 75 & $28.2 \pm 6.0$ \\
    \bottomrule
  \end{tabular}
\end{table}

The approach performs well in standard cases and shows robustness to variable lighting conditions. Performance degrades somewhat in partial view scenarios and significantly in the presence of brain shift, which is expected as our current approach assumes a rigid transformation between preoperative and intraoperative data.

\subsection{Comparison with State-of-the-Art Methods}

We compare our best approach with state-of-the-art methods for intraoperative registration on the clinical dataset. Table~\ref{tab:comparison} presents the results.

\begin{table}[htpb]
  \caption[Comparison with state-of-the-art methods]{Comparison with state-of-the-art methods for intraoperative registration.}\label{tab:comparison}
  \centering
  \begin{tabular}{l c c c}
    \toprule
      Method & TRE (mm) & Success Rate (\%) & Time (s) \\
    \midrule
      Point-based Registration & $3.25 \pm 0.65$ & 80 & $120.5 \pm 25.3$ \\
      Surface-based Registration & $2.87 \pm 0.57$ & 85 & $48.2 \pm 10.1$ \\
      Landmark-based Registration & $2.63 \pm 0.53$ & 88 & $95.7 \pm 20.1$ \\
      iNeRF Baseline \parencite{yen2020inerf} & $3.18 \pm 0.64$ & 79 & $22.4 \pm 4.7$ \\
      Cross-Modal Inverse NeRF \parencite{fehrentz2024intraoperative} & $2.37 \pm 0.47$ & 91 & $20.8 \pm 4.4$ \\
      \textbf{Our Approach} & $\mathbf{2.06 \pm 0.41}$ & \textbf{93} & $\mathbf{18.2 \pm 4.0}$ \\
    \bottomrule
  \end{tabular}
\end{table}

Our approach outperforms all baseline and state-of-the-art methods in terms of registration accuracy, success rate, and computational efficiency. Compared to the Cross-Modal Inverse NeRF method \parencite{fehrentz2024intraoperative}, our approach reduces the target registration error by approximately 13\% while also improving the success rate and reducing computation time.

\subsection{Ablation Studies}

We conduct ablation studies to understand the contribution of individual components to the overall performance. Table~\ref{tab:ablation} shows the results.

\begin{table}[htpb]
  \caption[Ablation studies]{Ablation studies showing the contribution of individual components.}\label{tab:ablation}
  \centering
  \begin{tabular}{l c c}
    \toprule
      Configuration & TRE (mm) & Success Rate (\%) \\
    \midrule
      Base Configuration (MI+L2, Gram Matrix, Multi-Stage) & $2.06 \pm 0.41$ & 93 \\
    \midrule
      \multicolumn{3}{l}{\textit{Loss Function Ablations}} \\
      Without MI Component & $2.35 \pm 0.47$ & 89 \\
      Without L2 Component & $2.18 \pm 0.44$ & 91 \\
    \midrule
      \multicolumn{3}{l}{\textit{Style Transfer Ablations}} \\
      Without Style Transfer & $3.04 \pm 0.61$ & 82 \\
      Using Y'UV Instead of Gram Matrix & $2.31 \pm 0.46$ & 90 \\
    \midrule
      \multicolumn{3}{l}{\textit{Optimization Strategy Ablations}} \\
      Using Joint Instead of Multi-Stage & $2.29 \pm 0.46$ & 90 \\
      Using Sequential Instead of Multi-Stage & $2.35 \pm 0.47$ & 88 \\
    \bottomrule
  \end{tabular}
\end{table}

The ablation studies confirm that each component contributes significantly to the overall performance. The style transfer component has the largest impact, followed by the MI component of the loss function. The optimization strategy also plays an important role, with the multi-stage approach providing the best performance.

\section{Computational Performance}

We analyze the computational performance of different approaches to assess their practical applicability in clinical settings. Table~\ref{tab:computational} presents the results.

\begin{table}[htpb]
  \caption[Computational performance of different registration approaches]{Computational performance of different registration approaches.}\label{tab:computational}
  \centering
  \begin{tabular}{l c c c c}
    \toprule
      Approach & Time (s) & GPU Memory (GB) & CPU Memory (GB) & FPS \\
    \midrule
      L2, No Style & $15.3 \pm 3.2$ & $2.8 \pm 0.6$ & $3.2 \pm 0.7$ & $18.5 \pm 3.9$ \\
      NCC, No Style & $16.7 \pm 3.5$ & $3.1 \pm 0.6$ & $3.4 \pm 0.7$ & $16.8 \pm 3.5$ \\
      MI, No Style & $19.2 \pm 4.0$ & $3.5 \pm 0.7$ & $3.7 \pm 0.8$ & $14.6 \pm 3.1$ \\
      L2, Y'UV Style & $17.2 \pm 3.6$ & $3.3 \pm 0.7$ & $3.5 \pm 0.7$ & $16.2 \pm 3.4$ \\
      MI+L2, Gram Matrix & $18.2 \pm 4.0$ & $3.8 \pm 0.8$ & $3.9 \pm 0.8$ & $15.3 \pm 3.2$ \\
    \bottomrule
  \end{tabular}
\end{table}

All approaches achieve real-time or near-real-time performance suitable for clinical use. The L2 loss without style transfer is the most computationally efficient but provides lower registration accuracy. Our recommended approach (MI+L2 with Gram matrix style transfer) requires approximately 18 seconds for registration, with moderate memory requirements.

\section{Summary of Findings}

Based on our comprehensive evaluation, we summarize the key findings:

\begin{enumerate}
    \item \textbf{Loss Functions}: The combination of Mutual Information and L2 loss provides the best performance for cross-modal registration, significantly outperforming the L2 baseline.
    
    \item \textbf{Style Transfer}: Gram matrix-based style transfer is the most effective method for bridging the appearance gap between preoperative and intraoperative images, improving registration accuracy by up to 55\%.
    
    \item \textbf{Optimization Strategy}: The multi-stage optimization approach offers the best balance between registration accuracy and computational efficiency.
    
    \item \textbf{Overall Performance}: Our best approach (MI+L2 loss with Gram matrix style transfer using multi-stage optimization) achieves a target registration error of $2.06 \pm 0.41$ mm on the clinical dataset, outperforming state-of-the-art methods.
    
    \item \textbf{Clinical Viability}: The approach demonstrates good performance across various clinical scenarios and achieves registration times suitable for intraoperative use.
\end{enumerate}

These results highlight the potential of enhanced NeRF-based registration for improving neurosurgical navigation through the exploration of alternative loss functions and style transfer techniques. 